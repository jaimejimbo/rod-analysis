%%%%%%%%%%%%%%%%%%%%%%%%%%%%%%%%%%%%%%%%%
% Journal Article
% LaTeX Template
% Version 1.3 (9/9/13)
%
% This template has been downloaded from:
% http://www.LaTeXTemplates.com
%
% Original author:
% Frits Wenneker (http://www.howtotex.com)
%
% License:
% CC BY-NC-SA 3.0 (http://creativecommons.org/licenses/by-nc-sa/3.0/)
%
%%%%%%%%%%%%%%%%%%%%%%%%%%%%%%%%%%%%%%%%%

%----------------------------------------------------------------------------------------
%	PACKAGES AND OTHER DOCUMENT CONFIGURATIONS
%----------------------------------------------------------------------------------------

\documentclass[twoside]{article}

\usepackage{lipsum} % Package to generate dummy text throughout this template

\usepackage[sc]{mathpazo} % Use the Palatino font
\usepackage[T1]{fontenc} % Use 8-bit encoding that has 256 glyphs
\linespread{1.05} % Line spacing - Palatino needs more space between lines
\usepackage{microtype} % Slightly tweak font spacing for aesthetics

\usepackage[hmarginratio=1:1,top=32mm,columnsep=20pt]{geometry} % Document margins
\usepackage{multicol} % Used for the two-column layout of the document
\usepackage[hang, small,labelfont=bf,up,textfont=it,up]{caption} % Custom captions under/above floats in tables or figures
\usepackage{booktabs} % Horizontal rules in tables
\usepackage{float} % Required for tables and figures in the multi-column environment - they need to be placed in specific locations with the [H] (e.g. \begin{table}[H])
\usepackage{hyperref} % For hyperlinks in the PDF

\usepackage{lettrine} % The lettrine is the first enlarged letter at the beginning of the text
\usepackage{paralist} % Used for the compactitem environment which makes bullet points with less space between them

\usepackage{abstract} % Allows abstract customization
\renewcommand{\abstractnamefont}{\normalfont\bfseries} % Set the "Abstract" text to bold
\renewcommand{\abstracttextfont}{\normalfont\small\itshape} % Set the abstract itself to small italic text

\usepackage{titlesec} % Allows customization of titles
\renewcommand\thesection{\Roman{section}} % Roman numerals for the sections
\renewcommand\thesubsection{\Roman{subsection}} % Roman numerals for subsections
\titleformat{\section}[block]{\large\scshape\centering}{\thesection.}{1em}{} % Change the look of the section titles
\titleformat{\subsection}[block]{\large}{\thesubsection.}{1em}{} % Change the look of the section titles

\usepackage{fancyhdr} % Headers and footers
\pagestyle{fancy} % All pages have headers and footers
\fancyhead{} % Blank out the default header
\fancyfoot{} % Blank out the default footer
\fancyhead[C]{Segregaci\'on granular en part\'iculas cil\'indricas$\bullet$ Diciembre 2015} % Custom header text
\fancyfoot[RO,LE]{\thepage} % Custom footer text

%----------------------------------------------------------------------------------------
%	TITLE SECTION
%----------------------------------------------------------------------------------------

\title{\vspace{-15mm}\fontsize{24pt}{10pt}\selectfont\textbf{Segregaci\'on granular en part\'iculas cil\'indricas}} % Article title

\author{
\large
\textsc{Jaime P\'erez Aparicio}\thanks{Enrique Velasco(direcci\'on, material, montaje) y Miguel Gonz\'alez Pinto(montaje experimental, c\'odigo, obtenci\'on de datos experimentales)}\\[2mm] % Your name
\normalsize Universidad Aut\'onoma de Madrid \\ % Your institution
\normalsize \href{mailto:jaime.pereza@estudiante.uam.es}{jaime.pereza@estudiante.uam.es} % Your email address
\vspace{-5mm}
}
\date{}

%----------------------------------------------------------------------------------------

\begin{document}

\maketitle % Insert title

\thispagestyle{fancy} % All pages have headers and footers

%----------------------------------------------------------------------------------------
%	ABSTRACT
%----------------------------------------------------------------------------------------

\begin{abstract}

\noindent En este experimento es estudiar la segregaci\'on granular que se produce en un sistema con dos tipos de part\'iculas cil\'indricas al hacerlas vibrar en una cuasimonocapa. Se miden distintas propiedades del sistema localmente (temperatura (energia cin\'etica), densidad y correlaci\'on angular) y se estudia su evoluci\'on temporal. La evoluci\'on temporal muestra como las part\'iculas de K(longitud/anchura) 14 se quedan en la zona superior de la imagen y adquieren una fase nem\'atica. Las peque\~nas (K4), con m\'as libertad de movimiento, van ''escapando'' de esa zona y adquieren m\'as energ\'ia cin\'etica. Para ellas s\'olo queda orden angular en la zona superior de la imagen. Se estudiar\'a c\'omo evoluciona el sistema temporalmente y como se comporta el sistema en la interfaz largas-cortas.

\end{abstract}

%----------------------------------------------------------------------------------------
%	ARTICLE CONTENTS
%----------------------------------------------------------------------------------------

\begin{multicols}{2} % Two-column layout throughout the main article text

\section*{Introducci\'on}

A veces, en sistemas aparentemente homog\'eneos aparecen ciertos estados estacionarios con patrones espaciales no triviales. El sistema ser\'ia homog\'eneo si las part\'iculas no interactuaran entre s\'i. Sin embargo, las part\'iculas chocan el\'asticamente entre s\'i debido a la energ\'ia que se introduce desde el exterior y, como no son iguales todas las part\'iculas, los choques no son iguales, por lo que el sistema no es homog\'eneo. Eso rompe la simetr\'ia espacial, por lo que se explican los patrones espaciales en los estados estacionarios.

En el estado estacionario se produce una segregaci\'on: las part\'iculas se separan seg\'un sus proporciones (caracterizadas por la constante $K=L/D$). Las part\'iculas m\'as alargadas quedan en fase nem\'atica en una zona, mientras que las menos alargadas quedan en la otra zona con m\'as libertad de movimiento y, por tanto, m\'as velocidad media (temperatura). Durante la fase estacionaria, la potencia introducida es igual a la disipada.

Pueden aparecer part\'iculas cortas donde est\'an las largas. Estar\'ian m\'as fr\'ias y ordenadas debido a la disipaci\'on, por lo que estar\'ian m\'as cerca las unas de las otras y podr\'ian entrar en fase nem\'atica o tetr\'atica. Cabe esperar que con el tiempo todas las part\'iculas cortas se separen de las largas.

Estos comportamientos est\'an causados por la forma de las part\'iculas. La energ\'ia que se introduce desde el exterior permite que las part\'iculas choquen entre s\'i. Por eso es importante tambi\'en la cantidad de energ\'ia que se introduce desde el exterior. La rapidez con la que el sistema se ordena depende de la energ\'ia que se introduzca, pero la forma en la que se ordena depende de los cocientes L/D. Si se suponen part\'iculas libres (pues no hay potenciales de interacci\'on entre ellas, aparentemente) se  define la energ\'ia libre como:

\begin{equation}
F = U - TS
\end{equation}

Se podr\'ia suponer que hay una ''entrop\'ia'' que el sistema maximiza con el tiempo. Esa entrop\'ia representar\'ia la interacci\'on entre part\'iculas (cuyo potencial de interacci\'on no se conoce). La interacci\'on podr\'ia suponerse como un escal\'on de Heaviside con un potencial muy alto (si es el\'astico ser\'ia infinito), pero es m\'as f\'acil meter la interacci\'on en la entrop\'ia. Adem\'as la interacci\'on depende de la temperatura.

Al estar las part\'iculas m\'as juntas, la energ\'ia se disipa hacia los bordes, pues se comportan como un s\'olido. Por ello su energ\'ia cin\'etica baja. Al chocar una part\'icula contra el grupo, su energ\'ia es absorbida y disipada hacia los bordes. En el video puede verse como aparece polvo cerca de los bordes. Puede deberse a que por esa zona se est\'a disipando m\'as energ\'ia.

%------------------------------------------------

\section*{Montaje}

Las part\'iculas se distribuyen en una cuasimonocapa. No es estrictamente una monocapa, pues las part\'iculas son cil\'indricas y pueden montarse un poco una encima de otra. Se podr\'ia haber hecho que la altura de la capa fuera igual al di\'ametro de las part\'iculas, pero entonces las part\'iculas no dispersar\'ian energ\'ia contra las tapas porque no chocar\'ian contra las tapas.

El sistema se hace vibrar con una m\'aquina especializada que permite elegir la amplitud y la frecuencia. A parte el sistema se ilumina con varias l\'amparas y se usa un papel semitransparente que difumina la luz para que los reflejos sean menores. Es importante que el sistema est\'e bien atornillado para que no se disipe energ\'ia por otros medios.

La c\'amara se coloca sobre el sistema y se le conecta un disparador autom\'atico que toma fotos en r\'afagas de 5. Las toma en 3 segundos, por lo que toma una foto cada 5/3 segundos. Luego se dispara una r\'afaga cada 15 minutos. Al principio del experimento se hacen un par de r\'afagas m\'as seguidas, pues el sistema al principio evoluciona m\'as r\'apidamente. El fondo del sistema es negro para aumentar el contraste y poder hacer mejor el an\'alisis de imagen.

%------------------------------------------------

\section*{Tratamiento de datos}

\subsection*{Imagej}

Las imagenes son tratadas con ''imagej''. Es un programa que permite obtener las posiciones y orientaciones de las part\'iculas. Mediante un script, escrito por Miguel, el programa aumenta el contraste, corta la imagen para que solo aparezca el c\'irculo en el que est\'an las part\'iculas y dibuja elipses encima de las part\'iculas. 

El programa devuelve el \'angulo al que se encuentra el eje largo. Tambi\'en devuelve los tama\~os de los ejes de la elipse y las longitudes de Feret (que son las distancias entre caras paralelas. En este caso la distancia m\'inima y la m\'axima. Tambi\'en develve la posici\'on media horizontal y vertical (la del centro de masas, en este caso, al haber simetr\'ia).

Sin embargo, pese a que las part\'iculas usadas tienen unas K bien definidas y medidas de 4 y 14, el programa devuelve elipses con K 5.5 y 17.5, por lo que hay cierto error al analizar las im\'agenes en cuanto a la obtenci\'on de su tama\~no. El script de python usa las K dadas por imagej, pues no se de d\'onde sale ese error y prefiero no forzar el uso de otro valor.

\subsection*{Script de python}

Los datos se pasan entonces a otro programa (que he escrito yo bas\'andome un poco en el que hizo Miguel). He usado un lenguaje distinto, as\'i que s\'olo he cogido un poco las ideas sobre los algoritmos que se usaban en el otro. He cambiado algunos de ellos porque no los entend\'ia o no me parec\'ian tan eficaces. Mi programa, al igual que el de Miguel, divide la distribuci\'on en c\'irculos del mismo radio, cubriendo toda la superficie disponible. En mi caso uso un radio m\'as grande, lo que suaviza un poco la distribuci\'on.

El programa es lento y posiblemente se pueda optimizar bastante (pues ahora usa una cantidad demasiado grande de RAM y de CPU), pero el tiempo es limitado y hay que poner prioridades. A\'un as\'i lo he optimizado bastante (evitando que se calculen varias veces las mismas cosas, usando algoritmos m\'as r\'apidos de ordenaci\'on, usando procesos y consiguiendo as\'i que se haga uso de m\'as procesadores...). Lo malo de evitar calcular las cosas varias veces es que hay que guardarlas mientras tanto, y eso hace que el uso de RAM sea exagerado. Posiblemente haya bugs que hagan que el uso se dispare y se podr\'ia disminuir en una cantidad notable, pero hay muchos objetivos y poco tiempo.

Python es un poco distinto a otros lenguajes de programaci\'on en cuanto a la liberaci\'on de memoria. Va realizando limpiezas peri\'odicas de objetos hu\'erfanos. Sin embargo se puede forzar esta limpieza para que se liberen recursos usados por objetos a los que no se puede acceder (porque se ha eliminado el puntero que les conecta al programa, por lo que solo est\'an malgastando memoria).

El programa hace los c\'alculos necesarios y saca los datos en forma de GIF animado, que luego otro programa llamado ffmpeg transforma a formato mp4, mucho m\'as comprimido y r\'apido. Hace gr\'aficas (para cada K) para las densidades locales, las correlaciones, las temperaturas, el \'area que cubren los clusters y un histograma del tama\~no de los clusters. En la del \'area que cubren los clusters, se representa \'area frente al tiempo, pudiendo ver as\'i el car\'acter de la evoluci\'on (exponencial, ley de potencias...).


%------------------------------------------------

\section*{Resultados}

\begin{table}[H]
\caption{Example table}
\centering
\begin{tabular}{llr}
\toprule
\multicolumn{2}{c}{Name} \\
\cmidrule(r){1-2}
First name & Last Name & Grade \\
\midrule
John & Doe & $7.5$ \\
Richard & Miles & $2$ \\
\bottomrule
\end{tabular}
\end{table}

\lipsum[5] % Dummy text

\begin{equation}
\label{eq:emc}
e = mc^2
\end{equation}

\lipsum[6] % Dummy text

%------------------------------------------------

\section{Discussion}

\subsection{Subsection One}

\lipsum[7] % Dummy text

\subsection{Subsection Two}

\lipsum[8] % Dummy text

%----------------------------------------------------------------------------------------
%	REFERENCE LIST
%----------------------------------------------------------------------------------------

\begin{thebibliography}{99} % Bibliography - this is intentionally simple in this template

\bibitem[Figueredo and Wolf, 2009]{Figueredo:2009dg}
Figueredo, A.~J. and Wolf, P. S.~A. (2009).
\newblock Assortative pairing and life history strategy - a cross-cultural
  study.
\newblock {\em Human Nature}, 20:317--330.
 
\end{thebibliography}

%----------------------------------------------------------------------------------------

\end{multicols}

\end{document}
