%%%%%%%%%%%%%%%%%%%%%%%%%%%%%%%%%%%%%%%%%
% Journal Article
% LaTeX Template
% Version 1.3 (9/9/13)
%
% This template has been downloaded from:
% http://www.LaTeXTemplates.com
%
% Original author:
% Frits Wenneker (http://www.howtotex.com)
%
% License:
% CC BY-NC-SA 3.0 (http://creativecommons.org/licenses/by-nc-sa/3.0/)
%
%%%%%%%%%%%%%%%%%%%%%%%%%%%%%%%%%%%%%%%%%

%----------------------------------------------------------------------------------------
%	PACKAGES AND OTHER DOCUMENT CONFIGURATIONS
%----------------------------------------------------------------------------------------

\documentclass[twoside]{article}

\usepackage{lipsum} % Package to generate dummy text throughout this template

\usepackage[sc]{mathpazo} % Use the Palatino font
\usepackage[T1]{fontenc} % Use 8-bit encoding that has 256 glyphs
\linespread{1.05} % Line spacing - Palatino needs more space between lines
\usepackage{microtype} % Slightly tweak font spacing for aesthetics

\usepackage[hmarginratio=1:1,top=32mm,columnsep=20pt]{geometry} % Document margins
\usepackage{multicol} % Used for the two-column layout of the document
\usepackage[hang, small,labelfont=bf,up,textfont=it,up]{caption} % Custom captions under/above floats in tables or figures
\usepackage{booktabs} % Horizontal rules in tables
\usepackage{float} % Required for tables and figures in the multi-column environment - they need to be placed in specific locations with the [H] (e.g. \begin{table}[H])
\usepackage{hyperref} % For hyperlinks in the PDF

\usepackage{lettrine} % The lettrine is the first enlarged letter at the beginning of the text
\usepackage{paralist} % Used for the compactitem environment which makes bullet points with less space between them

\usepackage{abstract} % Allows abstract customization
\renewcommand{\abstractnamefont}{\normalfont\bfseries} % Set the "Abstract" text to bold
\renewcommand{\abstracttextfont}{\normalfont\small\itshape} % Set the abstract itself to small italic text

\usepackage{titlesec} % Allows customization of titles
\renewcommand\thesection{\Roman{section}} % Roman numerals for the sections
\renewcommand\thesubsection{\Roman{subsection}} % Roman numerals for subsections
\titleformat{\section}[block]{\large\scshape\centering}{\thesection.}{1em}{} % Change the look of the section titles
\titleformat{\subsection}[block]{\large}{\thesubsection.}{1em}{} % Change the look of the section titles

\usepackage{fancyhdr} % Headers and footers
\pagestyle{fancy} % All pages have headers and footers
\fancyhead{} % Blank out the default header
\fancyfoot{} % Blank out the default footer
\fancyhead[C]{Segregaci\'on granular en part\'iculas cil\'indricas$\bullet$ Diciembre 2015} % Custom header text
\fancyfoot[RO,LE]{\thepage} % Custom footer text

%----------------------------------------------------------------------------------------
%	TITLE SECTION
%----------------------------------------------------------------------------------------

\title{\vspace{-15mm}\fontsize{24pt}{10pt}\selectfont\textbf{Segregaci\'on granular en part\'iculas cil\'indricas}} % Article title

\author{
\large
\textsc{Jaime P\'erez Aparicio}\thanks{A thank you or further information}\\[2mm] % Your name
\normalsize Universidad Aut\'onoma de Madrid \\ % Your institution
\normalsize \href{mailto:jaime.pereza@estudiante.uam.es}{jaime.pereza@estudiante.uam.es} % Your email address
\vspace{-5mm}
}
\date{}

%----------------------------------------------------------------------------------------

\begin{document}

\maketitle % Insert title

\thispagestyle{fancy} % All pages have headers and footers

%----------------------------------------------------------------------------------------
%	ABSTRACT
%----------------------------------------------------------------------------------------

\begin{abstract}

\noindent
El objetivo de este experimento es estudiar la segregaci\'on granular que se produce en un sistema con dos tipos de part\'iculas cil\'indricas al hacerlas vibrar en una cuasimonocapa. En el experimento se miden distintas propiedades del sistema localmente (temperatura (energia cin\'etica), densidad y correlaci\'on angular) y se estudia su evoluci\'on temporal. Se quiere ver como los rods con una relaci\'on longitud/distancia (K en adelante) mayor se acumulan en una zona bajando su temperatura y ''expulsando'' a las de K bajo, cuya energ\'ia cin\'etica se hace mayor. Tambi\'en las part\'iculas de K alto quedan ordenadas en una fase nem\'atica y la correlaci\'on angular de las de K bajo se anula, pr\'acticamente. Por tanto se alcanza un estado estacionario al dejar evolucionar al sistema durante un tiempo muy largo cuando la energ\'ia inyectada se compensa con la disipada.

\end{abstract}

%----------------------------------------------------------------------------------------
%	ARTICLE CONTENTS
%----------------------------------------------------------------------------------------

\begin{multicols}{2} % Two-column layout throughout the main article text

\section*{Introducci\'on}
 
\lipsum[2-3] % Dummy text

%------------------------------------------------

\section*{Montaje}

Las part\'iculas se distribuyen en una cuasimonocapa. No es estrictamente una monocapa, pues las part\'iculas son cil\'indricas y pueden montarse un poco una encima de otra. Se podr\'ia haber hecho que la altura de la capa fuera igual al di\'ametro de las part\'iculas, pero entonces las part\'iculas no dispersar\'ian energ\'ia contra las tapas porque no chocar\'ian contra las tapas. El sistema se hace vibrar con una m\'quina especializada que permite elegir la amplitud y la frecuencia. A parte el sistema se ilumina con varias l\'amparas y se usa un papel semitransparente que difumina la luz para que los reflejos sean menores. La c\'amara se coloca sobre el sistema y se le conecta un disparador autom\'atico que toma fotos en r\'afagas de 5. Las toma en 3 segundos, por lo que toma una foto cada 5/3 segundos. Luego se dispara una r\'afaga cada 15 minutos. Al principio del experimento se hacen un par de r\'afagas m\'as seguidas, pues el sistema al principio evoluciona m\'as r\'apidamente. El fondo del sistema es negro para aumentar el contraste y poder hacer mejor el an\'alisis de imagen.

\section{Methods}

Maecenas sed ultricies felis. Sed imperdiet dictum arcu a egestas. 
\begin{compactitem}
\item Donec dolor arcu, rutrum id molestie in, viverra sed diam
\item Curabitur feugiat
\item turpis sed auctor facilisis
\item arcu eros accumsan lorem, at posuere mi diam sit amet tortor
\item Fusce fermentum, mi sit amet euismod rutrum
\item sem lorem molestie diam, iaculis aliquet sapien tortor non nisi
\item Pellentesque bibendum pretium aliquet
\end{compactitem}
\lipsum[4] % Dummy text

%------------------------------------------------

\section{Results}

\begin{table}[H]
\caption{Example table}
\centering
\begin{tabular}{llr}
\toprule
\multicolumn{2}{c}{Name} \\
\cmidrule(r){1-2}
First name & Last Name & Grade \\
\midrule
John & Doe & $7.5$ \\
Richard & Miles & $2$ \\
\bottomrule
\end{tabular}
\end{table}

\lipsum[5] % Dummy text

\begin{equation}
\label{eq:emc}
e = mc^2
\end{equation}

\lipsum[6] % Dummy text

%------------------------------------------------

\section{Discussion}

\subsection{Subsection One}

\lipsum[7] % Dummy text

\subsection{Subsection Two}

\lipsum[8] % Dummy text

%----------------------------------------------------------------------------------------
%	REFERENCE LIST
%----------------------------------------------------------------------------------------

\begin{thebibliography}{99} % Bibliography - this is intentionally simple in this template

\bibitem[Figueredo and Wolf, 2009]{Figueredo:2009dg}
Figueredo, A.~J. and Wolf, P. S.~A. (2009).
\newblock Assortative pairing and life history strategy - a cross-cultural
  study.
\newblock {\em Human Nature}, 20:317--330.
 
\end{thebibliography}

%----------------------------------------------------------------------------------------

\end{multicols}

\end{document}
